\section{Tomography}

\subsection{State Tomography}

Notes taken from \cite{nielsen_quantum_2011}, only reworded and trimmed down for my benefit.
%% TODO: Clean up these notes at a later date to go in the introduction sections

In the classical world, characterising the dynamics of a system is trivial and known as
\textit{system identification}. The general idea is that we wish to know how the system behaves with
respect to any input, thus uniquely identifying it. In the quantum world, the analogue of this is
called \textit{quantum process tomography}. To understand process tomography, we must first
understand \textit{quantum state tomography}.

State tomography is the procedure of determining an unknown quantum state. This is harder than it
sounds: if we're given an unknown state $\rho$, we can't just measure the state and recover it
immediately way since measurement will \textit{disturb} the original state. In fact, \textit{there
is no quantum measurement which can distinguish non-orthogonal states with certainty}. However, if
we have an \textit{ensemble} of the same quantum state $\rho$, then it's possible to estimate
$\rho$.

If we represent the state of the system using its density matrix $\rho$, we may expand $\rho$ as
\begin{equation}
    \rho = \frac{\tr{\rho}I + \tr{X\rho}X + \tr{Y\rho}Y + \tr{Z\rho}Z}
                {2}
\end{equation}

Note that $\tr{Z\rho}$ can be interpreted as the \textit{expectation} of the observable $Z$.
Therefore, to estimate $\tr{Z\rho}$, we measure the observable $Z$ $m$-times to obtain outcomes
$z_1, \dots, z_m$ and calculate
\begin{equation}
    \tr{Z\rho} \approx \frac{1}{m} \sum_i^m z_i
\end{equation}
In general, this estimate is approximately a Gaussian with mean $\tr{Z\rho}$ and standard deviation
$\Delta(Z) / \sqrt{m}$, where $\Delta(Z)$ is the standard deviation of a single measurement. We can
apply this same method to estimate $\tr{X\rho}X$ and $\tr{Y\rho}Y$; with a large enough sample size
we obtain a good estimate for $\rho$. Additionally, since density matrices have unit trace, we know
that
\begin{equation}
    \tr{\rho} I = I
\end{equation}
This process can be generalised to a density matrix on $n$ qubits as
\begin{equation}
    \rho = \sum_{\vec{v}} \frac{\tr{\sigma_{v_1} \otimes \sigma_{v_2} \otimes \dots \otimes \sigma_{v_n}} \sigma_{v_1} \otimes \sigma_{v_2} \otimes \dots \otimes \sigma_{v_n}}{2^n}
\end{equation}
where $\vec{v} = \left(v_1, \dots, v_n\right)$ with entries $v_i$ chosen from the set $0, 1, 2, 3$,
i.e. each $\sigma_{v_i}$ is a particular Pauli matrix.

% TODO: look over this and possibly get an example? I get kind of what its doing but a concrete
% example would help

\subsection{Process Tomography}

To extend this notion to quantum process tomography is actually quite easy from a theoretical point
of view. If the state space of the system has $d$ dimensions ($d = 2$ for a single qubit), then we
choose $d^2$ pure quantum states $\ket{\psi_d}, \dots, \ket{\psi_{d^2}}$. The corresponding density
matrices $\ketbra{\psi_1}{\psi_1}, \dots, \ketbra{\psi_{d^2}}{\psi_{d^2}}$ for these states should
form a \textit{basis set} for the space of possible density matrices. 

Now, for each state $\ket{\psi_j}$, we prepare the system in that state and then subject it to the
process $\mathcal{E}(\ketbra{\psi_j}{\psi_j})$. Afterwards, we use state tomography to determine the
output state $\mathcal{E}(\ketbra{\psi_j}{\psi_j}$. Theoretically, this is all that we need to do,
since the matrices $\ketbra{\psi_j}{\psi_j}$ form a basis set, so any other possible density
matrices can be represented as a linear combination of the basis set. e.g.,
\begin{equation}
    \mathcal{E}(\ketbra{\Phi}{\Phi} + \ketbra{\Psi}{\Psi}) = 
    \mathcal{E}(\ketbra{\Phi}{\Phi}) + \mathcal{E}(\ketbra{\Psi}{\Psi})
\end{equation}
However, in reality we need to determine $\mathcal{E}$ from experimental data: operators are just a
theoretical tool, whereas experiments involve real numbers.
