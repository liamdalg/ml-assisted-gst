\section{Gate Set Tomography}

Notes taken from \cite{nielsen_gate_2020}.

\noindent \ac{GST} differs from state and process tomography in that:
\begin{itemize}
    \item It is almost entirely calibration-free. It does not depend upon a prior description of the
    measurements used (as in state tomography) or the states that can be prepared (as in process
    tomography). \textit{These are called ``reference frame'' operations in the literature?}
    \item It estimates an entire set of logic operations, rather than a single one.
\end{itemize}

\ac{GST} being calibration-free is incredibly important. Both state and process tomography are
limited in that they rely on \textit{accurate} characterisation of their ``reference frame''
operations. Typically, they're either unknown or misidentified.

\subsection{Mathematical Background}

% The important part for me :(

A quantum system is described by a $d$-dimensional \textit{Hilbert space} $\mathcal{H} =
\mathbb{C}^d$, where $d$ is the largest number of outcomes of a repeatable measurement. For a qubit,
$d = 2$. GST uses the \textit{Hilbert-Schmidt space}. The Hilbert-Schmidt space is the complex
$d^2$-dimensional vector space of $d \times d$ matrices. We're interested in the $d^2$-dimensional
subspace of Hermitian matrices, denoted $\mathcal{B(H)}$. The basis we use for $\mathcal{B(H)}$ is
the set of normalised Pauli matrices $\{ \mathbb{I} / \sqrt{2}, \sigma_x / \sqrt{2}, \sigma_y /
\sqrt{2}, \sigma_z / \sqrt{2} \}$. This basis has the following properties:
\begin{itemize}
    \item Hermicity: $B_i = B_i^{\dagger}$
    \item Orthonormality: $\tr{B_i B_j} = \delta_{ij}$
    \item Traceless for $i > 0$: $B_0 = \mathbb{I} / \sqrt{d}$ and $\tr{B_i} = 0 \enspace \forall{i}
    > 0$.
\end{itemize}

Elements of $\mathcal{B(H)}$ are represented using an `extension' of Dirac's bra-ket notation called
\textit{super bra-ket notation}. Some element $B$ is represented as a column vector $\superket{B}$,
and an element of its dual space by a row vector $\superbra{A}$. Everything works similarly to
regular Dirac notation, the main difference is that we can represent everything as vectors in
$\mathcal{B(H)}$ rather than the usual matrices.

Measurement of a quantum system yields an outcome from a set of $k$ possibilities. Therefore, the
$i$th outcome can be represented by a dual vector $\superbra{E_i}$, so that $\Pr(i | \rho) =
\superbraket{E_i | \rho} = \tr{E_i \rho}$. Since they represent probabilities, we require that $E_i
\ge 0$ and $\sum_i E_i = \mathbb{I}$. The $E_i$ are called \textit{effects}, and the set $\{E_i \}$
is called a \ac{POVM}.  Note that since both states and effects are both Hermitian, we can in fact
represent them in the $d^2$-dimensional real subspace of $\mathcal{B(H)}$. Therefore, any reference
to $\mathcal{B(H)}$ is referring to the real subspace.

\subsection{Quantum Logic Gates}

An \textit{ideal} quantum logic gate is \textit{reversible} and corresponds to a unitary transform
of $\mathcal{H}$. Such a gate would transform $\rho$ as $\rho \to U \rho U^{\dagger}$ for some
unitary matrix $U$. This is a linear transformation from $\mathcal{B(H)}$ to itself; the linear
transformation $\rho \to U \rho U^{\dagger}$ is called a \textit{superoperator}. In reality, logic
gates are not perfectly reversible. These superoperators are known as quantum processes or quantum
channels. We can represent any superoperator $\Lambda$ as a $d^2 \times d^2$ matrix, which acts on
$\superket{\rho} \in \mathcal{B(H)}$ by left multiplication. This representation is called the
\textit{transfer matrix} of $\Lambda$, and is denoted by $S_{\Lambda}$. Thus,
\begin{equation}
    \Lambda : \superket{\rho} \mapsto S_{\Lambda} \superket{\rho}
\end{equation}
If $\Lambda$ is performed on some input state $\rho$, then the probability of outcome $E_i$ is
therefore
\begin{equation}
    p_i = \superbraket{E_i | S_{\Lambda} | \rho} = \tr{E_i S_{\Lambda} \rho}
\end{equation}

Not all superoperators describe physically allowed operations. To be physically possible, they must
be:
\begin{itemize}
    \item \textit{Trace-preserving}: $\tr{\Lambda(\rho)}$ must equal 1 for all $\rho$.
    \item \textit{Completely Positive}: when $\Lambda$ acts on part of a larger system, it must
    preserve positivity for the entire system. A superoperator is \textit{positive} iff.
    $\Lambda(\rho) \ge 0$ for all $\rho$. A superoperator is \textit{completely positive} iff.
    $\Lambda \otimes \mathbb{I}_{\mathcal{A}}$ is positive for any auxiliary state space
    $\mathcal{A}$.
\end{itemize}
This \ac{CPTP} constraint alone is sufficient -- any \ac{CPTP} superoperator can be physically
implemented. The TP condition corresponds to $\superbra{\mathbb{I}} S_{\Lambda} =
\superbra{\mathbb{I}}$. Since our basis is traceless for $i > 0$, then $\Lambda$ is TP iff. the
first row of $S_{\Lambda}$ is $[1, 0, \dots, 0]$. The CP condition is a lot more tricky to describe.
We first rewrite $S_{\Lambda}$ in the operator-sum representation:
\begin{equation}
    \Lambda : \rho \mapsto \sum_{ij} \chi_{ij}^{\Lambda} B_i \rho B_j^{\dagger}
\end{equation}
where $\{B_i\}$ is a basis, and $\chi_{ij}^{\Lambda}$ is a matrix of coefficients called the ``Choi
process matrix'' which represents $\Lambda$. Similarly to the chi matrix representation from
earlier, this completely describes $\Lambda$ (read
\href{https://quantumcomputing.stackexchange.com/a/11814}{this answer} for more about their
relationship). The mapping between $S_{\Lambda}$ and $\chi_{ij}^{\Lambda}$ is known as the
Choi-Jamiołkowski isomorphism:
\begin{equation}
    \chi^{\Lambda} = d(S_{\Lambda} \otimes \mathbb{I}) \superket{\Pi_{\text{EPR}}}
\end{equation}
where $\superket{\Pi_{\text{EPR}}}$ is a maximally entangled state. Note that the equality above
really means element-wise equality in a consistent basis. This is all quite complex. Fortunately,
$\Lambda$ is CP iff. $\chi$ is positive semidefinite. 
\todoin{Research Choi representation vs. Chi representation}